% -------------------------------------------------------------
% file: mass1_test.tex
% -------------------------------------------------------------
% -------------------------------------------------------------
% Battelle Memorial Institute
% Pacific Northwest Laboratory
% -------------------------------------------------------------
% -------------------------------------------------------------
% Created December 13, 1999 by William A. Perkins
% Last Change: Sun Jan  8 15:43:13 2006 by William A. Perkins <perk@mcperktop.local>
% -------------------------------------------------------------
\documentclass[12pt,dvips,letterpaper]{article}

\usepackage[head=0.125in,margin=1.0in,top=0.5in,bottom=0.5in]{geometry}
\usepackage{mathptmx}
\usepackage{graphicx}
\usepackage{epsfig}
\usepackage[round,authoryear]{natbib}
\usepackage[hang]{caption}
\usepackage{dcolumn}
\usepackage[figuresright]{rotating}
\usepackage{color}

\title{MASS1 Validation Applications}

\begin{document}


\maketitle{}
\tableofcontents{}
% \appendix{}

\vspace{0.5in}

This document presents a series of simple tests that are used to check
the operation of MASS1. All necessary boundary condition and
configuration files are incorporated into the MASS1 source
repository, and so are distributed with the source.  These tests
check the basic function of  MASS1 hydrodynamics
(Section~\ref{sec:hydro-tests}) and contaminant transport
(Section~\ref{sec:transport-tests}).  

\section{Hydrodynamics Tests}
\label{sec:hydro-tests}


\subsection{Uniform Flow in a Rectangular Channel}
\label{sec:normal-flow}

%% This test was used to verify that MASS1 was able simulate 
%% uniform open channel flow and does so with multiple
%% computational mesh blocks.  The model was configured to simulate a
%% rectangular channel 3048~m (10,000~ft) long and 152.4~m (500~ft) wide
%% (Figure~\ref{fig:normal-channel}).  A normal depth of 1.524~m
%% (5.0~ft), an average velocity of 0.61~m/sec (2.0~ft/sec), and a Manning's
%% roughness coefficient of 0.026 were chosen.  The slope of the channel
%% required to produce these given conditions was predicted with
%% Manning's equation \citep{Chow59}:
%% \begin{displaymath}
%%   S_{o} = \left[ { V n } \over { d^{2 \over 3} } \right]^{2} ~ 
%%   = ~ 1.425 \times 10^{-4} ~ \textrm{m} \slash \textrm{m}
%% \end{displaymath}
%% Note that the bottom slope was computed using depth rather than
%% wetted perimeter.  This is consistent with the representation of
%% bottom friction in MASS2 (Section~\ref{sec:hydrodynamics-theory}),
%% where sidewall friction is not included.  The normal depth was applied
%% as the downstream boundary condition. A discharge of 141.6~m$^{3}$/sec
%% (5000 cfs) was applied at the upstream boundary.


This is a simple test in which MASS1 simulates normal flow conditions
in a simple channel.  The configuration consists of a single channel
link 3248~m (10656~ft) long having a rectangular cross section 30.48~m
(100~ft) wide.  A flow depth of 1.524~m (5.0~ft), average velocity of
0.609~m/s (2.0~ft/s), and a Manning's $n$ of 0.026 were chosen.  The
slope of the channel required to produce these conditions is given by
Manning's equation \citep{Chow59}:
\begin{displaymath}
  S_{o} = \left[ { V n } \over { d^{2 \over 3} } \right]^{2} ~ 
  = ~ 1.672 \times 10^{-4} ~ \textrm{m} \slash \textrm{m}
\end{displaymath}
or a rise of 1.782 feet over the length of the channel.  A total of 75
cross sections were used, spaced 144 feet apart. The simulation
results are shown in Figure~\ref{fig:normal-flow-results}.  The
initial elevation was set to 6.0 feet in the channel.  Boundary
conditions were a flow of 1000 cfs upstream and a stage of 5.0 feet
downstream. After a short time, a steady flow was established with a
depth of 5.0 feet, the normal depth, everywhere.

\begin{figure}[htbp]
  \begin{center}
    \input{channel.pstex_t}
    \caption{Channel configuration used for the uniform flow in a
      rectangular channel test.}
    \label{fig:normal-flow-channel}
  \end{center}
\end{figure}
\begin{figure}[htbp]
  \centering
  \includegraphics[width=3.0in]{../flow/normal/plot-elev.eps}
  \includegraphics[width=3.0in]{../flow/normal/plot.eps}
  \caption{Results of the uniform flow in a rectangular channel test
    showing the initial elevation (top) and depth and those for the
    final steady state conditions.}
  \label{fig:normal-flow-results}
\end{figure}

\subsection{Gradually Varied Flow in a Rectangular Channel}
\label{sec:subcr-norm-flow}

In this test, the channel used in Section~\ref{sec:normal-flow} was
extended to double its length in two ways
(Figure~\ref{fig:normal-slopebreak}).  In case 1, the extension was
placed downstream of the original channel.  The slope of this
extension was made steeper than the original channel, $S_{o} = 1.4208
\times 10^{-3}$, so that the expected flow would be subcritical, with a
depth of 0.762~m (2.5~ft), a velocity of 1.22~m/sec (4~ft/sec), and
a channel discharge of 141.6~m$^{3}$/sec (5000 cfs).  In case 2, the
steeper extension was placed upstream of the original channel.

\begin{figure}[tbph]
  \begin{center}
    \begin{tabular}[c]{m{3.0in}m{3.0in}}
    \multicolumn{1}{c}{Case 1} & \multicolumn{1}{c}{Case 2} \\
    \input{channel1.pstex_t} &    
    \input{channel2.pstex_t} \\
    \end{tabular}
    \caption{Dimensions of channel used for the gradually varying flow
      in a channel tests.}
    \label{fig:normal-slopebreak}
  \end{center}
\end{figure}

In both cases, normal depth in the downstream reach was simulated as
expected (Figure~\ref{fig:slopebreak-depth}).  In case~2, the expected
normal depth was also simulated in the upper reach. In case~1,
however, the upper reach was not long enough to obtain normal depth to
become established.  Instead, a classic M2 profile \citep{Chow59} was
simulated where depth was in transition to the shallower downstream
depth.  

\begin{figure}[htbp]
  \centering
  \begin{tabular}[b]{cc}
    Case 1 & Case 2 \\
    \includegraphics[width=3.0in]{../flow/slopebreak1/plot.eps} & 
    \includegraphics[width=3.0in]{../flow/slopebreak2/plot.eps} \\
    \includegraphics[width=3.0in]{../flow/slopebreak1/plot-elev.eps} &
    \includegraphics[width=3.0in]{../flow/slopebreak2/plot-elev.eps} \\
  \end{tabular}
  \caption{Simulated steady-state depths and water surface elevations
    from the gradually varied flow in a rectangular channel test.}
  \label{fig:slopebreak-depth}
\end{figure}


\subsection{Gradually Varied Flow in Prismatic Channels}
\label{sec:macdonald-1}

\cite{macdonald97:_analytic} presented analytic solutions to a set of
gradually and rapidly open channel flow problems.  MASS1 was
configured to simulate the two problems involving subcritical flow: problems
1 and 3. 

For test problem 1, MASS1 was configured to simulate a
rectangular channel 1000~m (3281~ft) long and 10~m (3.281~ft) wide
having a slope computed according to
\begin{equation}
  S_{o}\left( x \right) ~ = ~ \left[ 1 - \frac{4}{g d^{3}(x)} \right]
  d'\left( x \right) 
  + ~ 0.36 \frac{\left[ 2 d(x) + 10 \right]^{4 \slash 3}}{\left[10
  d(x) \right]^{10 \slash 3}} \nonumber  
\end{equation}
where slope, $S_{o}$, and depth, $d$, are functions of distance along
the channel, $x$, and $g$ is gravitational acceleration.  The depth
and its derivative is given by
\begin{equation}
  \label{eq:macdonald-1-depth}
  d \left( x \right) ~ = ~ \left( \frac{4}{g} \right)^{1 \slash 3} \left\{ 1 +
  \frac{1}{2} \exp \left[ -16 \left( \frac{x}{1000} - \frac{1}{2}
  \right)^{2} \right] \right\} 
\end{equation}
and 
\begin{equation}
  d' \left( x \right) ~ = ~ - \left( \frac{4}{g} \right)^{1 \slash
  3} \frac{2}{125} \left(  \frac{x}{1000} - \frac{1}{2} \right) \exp
  \left[ -16 \left( \frac{x}{1000} - \frac{1}{2} 
  \right)^{2} \right] \nonumber
\end{equation}
A single link with 101 cross sections was used resulting in a cross
section spacing of 10~m (3.281~ft).  MASS1 was started with stage of
7.62~m (25~ft). A discharge of 20~m$^{3}$/sec (707.3~cfs) was imposed
at the upstream boundary and a stage of 0.748~m (2.45~ft) was imposed
at the downstream boundary. The simulated steady-state depth was very
close to the expected, as shown in Figure~\ref{fig:macdonald-1}.

\begin{figure}[htbp]
  \centering
  \includegraphics[width=3.0in]{../flow/MacDonald-1/plot-elev.eps}
  \includegraphics[width=3.0in]{../flow/MacDonald-1/plot-depth.eps}
  \caption{Simulated water surface elevation and depth compared with the
    analytic solution to problem~3 of \cite{macdonald97:_analytic}.}
  \label{fig:macdonald-1}
\end{figure}

For test problem 3, MASS1 was configured to simulate a channel 5~km
(3.1~miles) long having a Manning roughness coefficient of 0.030 and a
trapezoidal section with a base width of 10~m (32.8~ft) and a 1:2 side
slope.  The channel discharge was 20~m$^{3}$/sec (707.3~cfs) and the
downstream depth was 1.125~m (3.69~ft).  The bottom slope was given by
\begin{equation}
  S_{o} ~ = ~ \left\{ 1 - { { 400 \left[ 10 + 4d(x) \right] } \over { g
  \left[ 10 + 2d(x) \right]^{3} d(x)^{3} } } \right\} d'(x)
  ~ + ~ 0.36 { { \left[ 10 + 2d(x)\sqrt{5} \right]^{4 \slash 3} } \over {
  \left[ 10 + 2d(x) \right]^{10 \slash 3} d(x)^{10 \slash 3} } } \nonumber
\end{equation}
where 
\begin{equation}
  \label{eq:mac-3-d}
  d(x) ~ = ~ \frac{9}{8} + \frac{1}{4} \sin \left(\frac{\pi x}{500}\right)
\end{equation}
and
\begin{equation}
  d'(x) ~ = ~ \frac{\pi}{2000} \cos \left(\frac{\pi x}{500}\right) \nonumber
\end{equation}
The solution to the problem is given by equation~\ref{eq:mac-3-d}.
A single link with a cross section spacing of 25~m (82.0~ft) was
used.  An initial stage of 

Depths simulated by MASS1 were consistent with the analytic solution
(Figure~\ref{fig:macdonald-3-results}).  

\begin{figure}[hbtp]
  \centering
  \includegraphics[width=3.0in]{../flow/MacDonald-3/plot-elev.eps}
  \includegraphics[width=3.0in]{../flow/MacDonald-3/plot-depth.eps}  
  \caption{Simulated water surface elevation and depth compared with the
    analytic solution to problem~3 of \cite{macdonald97:_analytic}.}
  \label{fig:macdonald-3-results}
\end{figure}

\subsection{Lateral Inflow and Outflow}
\label{sec:test-lateral}

This is another simple test used to make sure lateral inflow is
handled appropriately by MASS1.  The channel configuration in
Section~\ref{sec:normal-flow} was used. A held flow of 1000 cfs was
applied at the upstream boundary condition and a held stage of 5.0
feet was applied downstream.  A constant lateral inflow of 500 cfs was
applied over the entire length of the channel ($500 \slash 10656 = 4.692
\times 10^{-2}$ cfs/ft).  In addition, a constant lateral
\textit{outflow} of 500 cfs was applied as a separate case.  Profiles
of simulated discharge for both the inflow and outflow cases are shown
in Figure~\ref{fig:test-lateral}.  As expected, discharge linearly
increased/decreased along the channel to the appropriate channel
outflow: 1500 cfs in the inflow case, 500 cfs in the outflow case.

\begin{figure}[htbp]
  \begin{center}
    \includegraphics[width=4.5in]{../flow/lateral/plot.eps}
    \caption{Resulting simulated discharge in lateral inflow and
      outflow tests.}  
    \label{fig:test-lateral}
  \end{center}
\end{figure}

\subsection{Channel Storage}
\label{sec:test-storage}

The channel storage test exercises the use of a downstream flow
boundary condition by following total channel storage while the
channel inflow and outflow changes.  The configuration in
Section~\ref{sec:normal-flow} was brought to steady state, then the
downstream boundary was switched to a flow condition, and a 1000 cfs
flow applied.  

After as short time, a 1-hour flow pulse of 400 cfs was applied
upstream (see Figure~\ref{fig:test-storage-inflow}). After enough time
for the channel to reach a steady condition, and outflow pulse, of the
same magnitude as the inflow pulse, was applied downstream.  For each
simulation time step, channel storage was computed by estimating the
volume between each of the cross sections and summing.  

This test was also performed using a lateral inflow pulse.  The pulse
used was the same size, 400 cfs, but spread over the length of the
entire channel ($400 \slash 10656 = 3.754 \times 10^{-2}$
cfs/ft). Those results are shown in
Figure~\ref{fig:test-storage-lateral}. 


\begin{figure}[htbp]
  \begin{center}
    \includegraphics[width=4.5in]{../flow/storage/plot-flowbc.eps}
    \caption{Results of channel storage test with varying upstream inflow.}
    \label{fig:test-storage-inflow}
  \end{center}
\end{figure}

\begin{figure}[htbp]
  \begin{center}
    \includegraphics[width=4.5in]{../flow/storage/plot-lateral.eps}
    \caption{Results of channel storage test with varying lateral inflow.}
    \label{fig:test-storage-lateral}
  \end{center}
\end{figure}


\section{Transport Tests}
\label{sec:transport-tests}


\subsection{Pure Advection}
\label{sec:test-advection-1-link}

This tests the transport of an arbitrary contaminant by advection
only.  It exercises the scalar transport capabilities of MASS1 and
highlights some benefits of the underlying transport algorithm.  

The channel configuration shown in Section~\ref{sec:normal-flow} was
used, but the cross section spacing was halved; a total of 149 cross
sections were spaced at 72 feet.  As in Section~\ref{sec:normal-flow},
a flow of 1000 cfs and downstream stage of 5.0 feet was imposed. A pulse
of contaminant concentration, shown in
Figure~\ref{fig:test-advection-bc}, was used as a boundary condition,
which could be described as
\begin{equation}
  \label{eq:advection-bc-conc}
  C_{o}(t) = \left\{ 
    \begin{array}[c]{cl}
      0 & \mathrm{for} ~ t \le 0, \\
      C_{p} \left( 1 - { { t_{p} - t } \over { t_{p} } } \right) &
      \mathrm{for} ~ 0 < t \le t_{p}, \\
      C_{p} \left( 1 - { { t - t_{p} } \over { t_{p} } } \right) &
      \mathrm{for} ~ t_{p} < t \le 2 t_{p}, \\
      0 & \mathrm{for} ~ t > 2 t_{p}
    \end{array} 
  \right.   
\end{equation}
where $t$ is time measured from start of the pulse (06:00 in
Figure~\ref{fig:test-advection-bc}), $t_{p}$ is the time to the peak
(24 min), and $C_{p}$ is the peak concentration (10.0).  With pure
advection, this pulse should be translated downstream without change in
shape.  The translated concentration is
\[
  C(x, t) ~ = ~ C_{o}(t - x/v)
\]
where $x$ is the distance along the channel and $v$ is the flow
velocity (2.0 ft/s in this case).  

MASS1 should translate the contaminant pulse provided the courant
number, $C_{n}$,
\begin{equation}
  \label{eq:courant-number}
  C_{n} = { { v \Delta t } \over { \Delta x } }
\end{equation}
is equal to 1, where $\Delta t$ is the computational time step, and
$\Delta x$ is the cross section spacing (72 feet). The transport time
step for the simulation would need to be
\[
\Delta t = { { C_{n} \Delta x } \over v } ~ = ~ {{ 1.0 \times 72.0 ~
    \mathrm{ft}} \over {2.0 ~ \mathrm{ft} \slash \mathrm{s} }} ~ = ~
    36 ~\mathrm{s} ~ = ~ 0.01 ~ \mathrm{hr}
\]
The results of the simulation with this time step are shown in the
upper graph of Figure~\ref{fig:test-advection-1-results}.  To examine
the effect of a lower $C_{n}$, another simulation was performed with a
transport time step of 0.001 hour, meaning a courant number of 0.1.
The results of this simulation are shown in the lower graph of
Figure~\ref{fig:test-advection-1-results}.  With a lower courant
number, the pulse becomes smoothed due to numerical diffusion.  

\begin{figure}[htbp]
  \begin{center}
    \includegraphics[width=3.0in]{../transport/advection-1-segment/bcplot.eps}
    \caption{Upstream concentration boundary condition used for
      advection tests.} 
    \label{fig:test-advection-bc}
  \end{center}
\end{figure}

\begin{figure}[htbp]
  \begin{center}
    \includegraphics[width=3.0in]{../transport/advection-1-segment/plot-Cn=1.0.eps}
    \includegraphics[width=3.0in]{../transport/advection-1-segment/plot-Cn=0.1.eps}
    \caption{Results for the advection through a single link test;
      simulated concentration is compared to translations of the
      boundary condition.} 
    \label{fig:test-advection-1-results}
  \end{center}
\end{figure}

\begin{figure}[htbp]
  \begin{center}
    \includegraphics[width=3.0in]{../transport/advection-2-segment/plot-Cn=1.0.eps}
    \includegraphics[width=3.0in]{../transport/advection-2-segment/plot-Cn=0.1.eps}
    \caption{Results for the test of advection through multiple links;
      simulated concentration is compared to translations of the upstream
      boundary condition.}
    \label{fig:test-advection-2-results}
  \end{center}
\end{figure}


\subsection{Advective Diffusion}
\label{sec:test-advection-diffusion-1-link}

This test compares the MASS1 transport simulation to an analytic
solution to the advection-diffusion equation \citep[][\S~2.4]{Fis79}.
The analytic solution describes the transport of a sharp concentration
front through a medium moving at a constant velocity and having an
initial concentration of zero.  The concentration along the channel is
described as
\begin{equation}
  \label{eq:2}
  C(x,t) = { { C_{o} } \over 2 } \left[ { 1 - \mathrm{erf} \left( { { x - vt } 
          \over { \sqrt{4 K_{T} t} } } \right) } \right]
\end{equation}
where $t$ is time, $x$ is the distance along the channel, $v$ is the
flow velocity, $C_{o}$ is the height of the concentration front, and
$K_{T}$ is the longitudinal dispersion coefficient.  

The same channel configuration as
Section~\ref{sec:test-advection-1-link} was used to simulate this
situation, with a $K_{T}$ of 30.0.  The channel was given an initial
concentration of zero, and a constant concentration of 10.0. Also, as
in Section~\ref{sec:test-advection-1-link}, two transport time steps
were used: one yielding a courant number of 1.0, the other 0.1.
Figure~\ref{fig:test-advection-diffusion-1-results} shows the progress
of the simulated concentration front down the channel at various
times, and compares it to (\ref{eq:2}).


\begin{figure}[htbp]
  \begin{center}
    \includegraphics[width=3.0in]{../transport/advection-diffusion-1-segment/plot-Cn=1.0.eps}
    \includegraphics[width=3.0in]{../transport/advection-diffusion-1-segment/plot-Cn=0.1.eps}
    \caption{Results for the test of advective diffusion through a single link;
      simulated concentration is compared to an analytical solution.}
    \label{fig:test-advection-diffusion-1-results}
  \end{center}
\end{figure}


\begin{figure}[htbp]
  \begin{center}
    \includegraphics[width=4.5in]{../transport/advection-diffusion-2-segment/plot-Cn=1.0.eps}
    \includegraphics[width=4.5in]{../transport/advection-diffusion-2-segment/plot-Cn=0.1.eps}
    \caption{Results for the test of advective diffusion through multiple links;
      simulated concentration is compared to an analytical solution.}
    \label{fig:test-advection-diffusion-2-results}
  \end{center}
\end{figure}


\subsection{Transport with Lateral Inflow/Outflow}
\label{sec:test-transport-lateral}

In the current MASS1 version, lateral inflow is considered to have the
same gas concentration and temperature as the water in the main
channel.  This will probably change, so that concentrations and
temperatures of lateral inflow can be specified.  This test is used to
verify that when there is lateral inflow/outflow, it does not affect
the simulated channel concentrations.  

Using the channel configuration of
Section~\ref{sec:test-advection-1-link}, a lateral inflow of 500 cfs
was applied along the entire length of channel ($4.692 \times 10^{-2}
cfs/ft$), and a constant inflow concentration (10.0) was applied
upstream.  This was allowed to come to steady state, with the
expectation that the concentration would remain at 10.0 throughout the
channel.  The results are shown in
Figure~\ref{fig:test-tranport-lateral-inflow}, which show that the
concentration did indeed remain constant. 

A similar simulation was performed with a lateral \textit{outflow} of
500 cfs applied along the entire length of channel.  The results of
this simulation are shown in
Figure~\ref{fig:test-tranport-lateral-outflow}.

\begin{figure}[htbp]
  \begin{center}
    \includegraphics[width=5in]{../transport/lateral-inflow/plot-inflow.eps}
    \caption{Results for test of transport with lateral inflow;
      concentration remains constant because the concentration of
      lateral inflow is assumed to be that in the channel.} 
    \label{fig:test-tranport-lateral-inflow}
  \end{center}
\end{figure}

\begin{figure}[htbp]
  \begin{center}
    \includegraphics[width=4.5in]{../transport/lateral-inflow/plot-outflow.eps}
    \caption{Results for test of transport with lateral outflow.} 
    \label{fig:test-tranport-lateral-outflow}
  \end{center}
\end{figure}

% \begin{figure}[htbp]
%   \begin{center}
%     \includegraphics[width=4.5in]{../transport/lateral-inflow/bcplot.eps}
%     \caption{Total lateral inflow used to test transport with varying
%       lateral contribution.} 
%     \label{fig:test-transport-lateral-flow}
%   \end{center}
% \end{figure}

% \begin{figure}[htbp]
%   \begin{center}
%     \includegraphics[width=4.5in]{../transport/lateral-inflow/plot-vary.eps}
%     \caption{Results for test of transport with varying lateral
%       contribution.}  
%     \label{fig:test-tranport-lateral-vary}
%   \end{center}
% \end{figure}

\bibliographystyle{pnnl}
\bibliography{mass1test}


\end{document}


%%% Local Variables: 
%%% mode: latex
%%% TeX-master: t
%%% End: 
