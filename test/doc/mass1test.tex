% -------------------------------------------------------------
% file: mass1_test.tex
% -------------------------------------------------------------
% -------------------------------------------------------------
% Battelle Memorial Institute
% Pacific Northwest Laboratory
% -------------------------------------------------------------
% -------------------------------------------------------------
% Created December 13, 1999 by William A. Perkins
% Last Change: Thu Jan  5 08:27:58 2006 by William A. Perkins <perk@McPerk.pnl.gov>
% -------------------------------------------------------------
\documentclass[12pt,dvips,letterpaper]{article}

\usepackage[head=0.125in,margin=1.0in,top=0.5in,bottom=0.5in]{geometry}
\usepackage{mathptmx}
\usepackage{graphicx}
\usepackage{epsfig}
\usepackage[round,authoryear]{natbib}
\usepackage[hang]{caption}
\usepackage{dcolumn}
\usepackage[figuresright]{rotating}
\usepackage{color}

\title{MASS1 Test Applications}

\begin{document}


\maketitle{}
\tableofcontents{}
% \appendix{}


This section presents a series of simple tests that are used to check
the operation of MASS1. All necessary boundary condition and
configuration files are incorporated into the MASS1 source
repository, and so are distributed with the source.  These tests
check the basic functions of  MASS1: hydrodynamics
(Section~\ref{sec:hydro-tests}) and contaminant transport
(Section~\ref{sec:transport-tests}).  

\section{Hydrodynamics Tests}
\label{sec:hydro-tests}


\subsection{Normal Flow}
\label{sec:normal-flow}

This is a simple test in which MASS1 simulates normal flow conditions
in a simple channel.  The configuration consists of a single channel
link having a constant rectangular cross section with the following
dimensions (see also Figure:
\begin{itemize}
\item 10656 feet long;
\item bottom width of 100.0 feet;
\item 5.0 feet flow depth; and
\item Manning's $n$ of 0.026.
\end{itemize}
A total of 75 cross sections were used, spaced 144 feet apart. If a
velocity of 2.0 feet/second, corresponding to a flow of 1000.0 cfs, is
assumed, the channel slope would be $1.672 \times 10^{-4}$ feet/foot,
or a rise of 1.724 feet over the length of the channel.  

The test results are shown in Figure~\ref{fig:normal-flow-results}.
The initial elevation was set to 6.0 feet in the channel.  Boundary
conditions were held flow of 1000 cfs upstream and held stage of 5.0
feet downstream. After a short time, a steady flow was established
with a depth of 5.0 feet, the normal depth, everywhere.  

\begin{figure}[htbp]
  \begin{center}
    \input{channel.pstex_t}
    \caption{Channel configuration used in the normal flow test.}
    \label{fig:normal-flow-channel}
  \end{center}
\end{figure}
\begin{figure}[htbp]
  \begin{center}
    \includegraphics[width=4.5in]{../flow/normal/plot-elev.eps}
    \includegraphics[width=4.5in]{../flow/normal/plot.eps}
    \caption{Results of the normal flow test showing the initial
      elevation (top) and depth and those for the final steady state
      conditions.}
    \label{fig:normal-flow-results}
  \end{center}
\end{figure}


\subsection{Gradually Varied Subcritical Flow in a Rectangular Channel}
\label{sec:subcr-norm-flow}

\begin{figure}[htbp]
  \centering
  \begin{tabular}[b]{cc}
    Case 1 & Case 2 \\
    \includegraphics[width=3.0in]{../flow/slopebreak1/plot.eps} & 
    \includegraphics[width=3.0in]{../flow/slopebreak2/plot.eps} \\
    \includegraphics[width=3.0in]{../flow/slopebreak1/plot-elev.eps} &
    \includegraphics[width=3.0in]{../flow/slopebreak2/plot-elev.eps} \\
  \end{tabular}
  \caption{Simulated steady-state depths and water surface elevations
    from the gradually varied flow in a rectangular channel test.}
  \label{fig:slopebreak-depth}
\end{figure}


\subsection{Lateral Inflow and Outflow}
\label{sec:test-lateral}

This is another simple test used to make sure lateral inflow is
handled appropriately by MASS1.  The channel configuration in
Section~\ref{sec:normal-flow} was used. A held flow of 1000 cfs was
applied at the upstream boundary condition and a held stage of 5.0
feet was applied downstream.  A constant lateral inflow of 500 cfs was
applied over the entire length of the channel ($500 \slash 10656 = 4.692
\times 10^{-2}$ cfs/ft).  In addition, a constant lateral
\textit{outflow} of 500 cfs was applied as a separate case.  Profiles
of simulated discharge for both the inflow and outflow cases are shown
in Figure~\ref{fig:test-lateral}.  As expected, discharge linearly
increased/decreased along the channel to the appropriate channel
outflow: 1500 cfs in the inflow case, 500 cfs in the outflow case.

\begin{figure}[htbp]
  \begin{center}
    \includegraphics[width=4.5in]{../flow/lateral/plot.eps}
    \caption{Resulting simulated discharge in lateral inflow and
      outflow tests.}  
    \label{fig:test-lateral}
  \end{center}
\end{figure}

\subsection{Channel Storage}
\label{sec:test-storage}

The channel storage test exercises the use of a downstream flow
boundary condition by following total channel storage while the
channel inflow and outflow changes.  The configuration in
Section~\ref{sec:normal-flow} was brought to steady state, then the
downstream boundary was switched to a flow condition, and a 1000 cfs
flow applied.  

After as short time, a 1-hour flow pulse of 400 cfs was applied
upstream (see Figure~\ref{fig:test-storage-inflow}). After enough time
for the channel to reach a steady condition, and outflow pulse, of the
same magnitude as the inflow pulse, was applied downstream.  For each
simulation time step, channel storage was computed by estimating the
volume between each of the cross sections and summing.  

This test was also performed using a lateral inflow pulse.  The pulse
used was the same size, 400 cfs, but spread over the length of the
entire channel ($400 \slash 10656 = 3.754 \times 10^{-2}$
cfs/ft). Those results are shown in
Figure~\ref{fig:test-storage-lateral}. 


\begin{figure}[htbp]
  \begin{center}
    \includegraphics[width=4.5in]{../flow/storage/plot-flowbc.eps}
    \caption{Results of channel storage test with varying upstream inflow.}
    \label{fig:test-storage-inflow}
  \end{center}
\end{figure}

\begin{figure}[htbp]
  \begin{center}
    \includegraphics[width=4.5in]{../flow/storage/plot-lateral.eps}
    \caption{Results of channel storage test with varying lateral inflow.}
    \label{fig:test-storage-lateral}
  \end{center}
\end{figure}


\section{Transport Tests}
\label{sec:transport-tests}


\subsection{Pure Advection}
\label{sec:test-advection-1-link}

This tests the transport of an arbitrary contaminant by advection
only.  It exercises the scalar transport capabilities of MASS1 and
highlights some benefits of the underlying transport algorithm.  

The channel configuration shown in Section~\ref{sec:normal-flow} was
used, but the cross section spacing was halved; a total of 149 cross
sections were spaced at 72 feet.  As in Section~\ref{sec:normal-flow},
a flow of 1000 cfs and downstream stage of 5.0 feet was imposed. A pulse
of contaminant concentration, shown in
Figure~\ref{fig:test-advection-bc}, was used as a boundary condition,
which could be described as
\begin{equation}
  \label{eq:advection-bc-conc}
  C_{o}(t) = \left\{ 
    \begin{array}[c]{cl}
      0 & \mathrm{for} ~ t \le 0, \\
      C_{p} \left( 1 - { { t_{p} - t } \over { t_{p} } } \right) &
      \mathrm{for} ~ 0 < t \le t_{p}, \\
      C_{p} \left( 1 - { { t - t_{p} } \over { t_{p} } } \right) &
      \mathrm{for} ~ t_{p} < t \le 2 t_{p}, \\
      0 & \mathrm{for} ~ t > 2 t_{p}
    \end{array} 
  \right.   
\end{equation}
where $t$ is time measured from start of the pulse (06:00 in
Figure~\ref{fig:test-advection-bc}), $t_{p}$ is the time to the peak
(24 min), and $C_{p}$ is the peak concentration (10.0).  With pure
advection, this pulse should be translated downstream without change in
shape.  The translated concentration is
\[
  C(x, t) ~ = ~ C_{o}(t - x/v)
\]
where $x$ is the distance along the channel and $v$ is the flow
velocity (2.0 ft/s in this case).  

MASS1 should translate the contaminant pulse provided the courant
number, $C_{n}$,
\begin{equation}
  \label{eq:courant-number}
  C_{n} = { { v \Delta t } \over { \Delta x } }
\end{equation}
is equal to 1, where $\Delta t$ is the computational time step, and
$\Delta x$ is the cross section spacing (72 feet). The transport time
step for the simulation would need to be
\[
\Delta t = { { C_{n} \Delta x } \over v } ~ = ~ {{ 1.0 \times 72.0 ~
    \mathrm{ft}} \over {2.0 ~ \mathrm{ft} \slash \mathrm{s} }} ~ = ~
    36 ~\mathrm{s} ~ = ~ 0.01 ~ \mathrm{hr}
\]
The results of the simulation with this time step are shown in the
upper graph of Figure~\ref{fig:test-advection-1-results}.  To examine
the effect of a lower $C_{n}$, another simulation was performed with a
transport time step of 0.001 hour, meaning a courant number of 0.1.
The results of this simulation are shown in the lower graph of
Figure~\ref{fig:test-advection-1-results}.  With a lower courant
number, the pulse becomes smoothed due to numerical diffusion.  

\begin{figure}[htbp]
  \begin{center}
    \includegraphics[width=4.0in]{../transport/advection-1-segment/bcplot.eps}
    \caption{Upstream concentration boundary condition used for
      advection tests.} 
    \label{fig:test-advection-bc}
  \end{center}
\end{figure}

\begin{figure}[htbp]
  \begin{center}
    \includegraphics[width=5.0in]{../transport/advection-1-segment/plot-Cn=1.0.eps}
    \includegraphics[width=5.0in]{../transport/advection-1-segment/plot-Cn=0.1.eps}
    \caption{Results for the advection through a single link test;
      simulated concentration is compared to translations of the
      boundary condition.} 
    \label{fig:test-advection-1-results}
  \end{center}
\end{figure}

% \clearpage{}

% \subsection{Advection through Multiple Links}
% \label{sec:test-advection-2-link}

% \begin{figure}[htbp]
%   \begin{center}
%     \includegraphics[width=4.5in]{../transport/advection-2-segment/plot-Cn=1.0.eps}
%     \includegraphics[width=4.5in]{../transport/advection-2-segment/plot-Cn=0.1.eps}
%     \caption{Results for the test of advection through multiple links;
%       simulated concentration is compared to translations of the upstream
%       boundary condition.}
%     \label{fig:test-advection-2-results}
%   \end{center}
% \end{figure}

% \clearpage{}


\subsection{Advective Diffusion}
\label{sec:test-advection-diffusion-1-link}

This test compares the MASS1 transport simulation to an analytic
solution to the advection-diffusion equation \citep[][\S~2.4]{Fis79}.
The analytic solution describes the transport of a sharp concentration
front through a medium moving at a constant velocity and having an
initial concentration of zero.  The concentration along the channel is
described as
\begin{equation}
  \label{eq:2}
  C(x,t) = { { C_{o} } \over 2 } \left[ { 1 - \mathrm{erf} \left( { { x - vt } 
          \over { \sqrt{4 K_{T} t} } } \right) } \right]
\end{equation}
where $t$ is time, $x$ is the distance along the channel, $v$ is the
flow velocity, $C_{o}$ is the height of the concentration front, and
$K_{T}$ is the longitudinal dispersion coefficient.  

The same channel configuration as
Section~\ref{sec:test-advection-1-link} was used to simulate this
situation, with a $K_{T}$ of 30.0.  The channel was given an initial
concentration of zero, and a constant concentration of 10.0. Also, as
in Section~\ref{sec:test-advection-1-link}, two transport time steps
were used: one yielding a courant number of 1.0, the other 0.1.
Figure~\ref{fig:test-advection-diffusion-1-results} shows the progress
of the simulated concentration front down the channel at various
times, and compares it to (\ref{eq:2}).


\begin{figure}[htbp]
  \begin{center}
    \includegraphics[width=4.5in]{../transport/advection-diffusion-1-segment/plot-Cn=1.0.eps}
    \includegraphics[width=4.5in]{../transport/advection-diffusion-1-segment/plot-Cn=0.1.eps}
    \caption{Results for the test of advective diffusion through a single link;
      simulated concentration is compared to an analytical solution.}
    \label{fig:test-advection-diffusion-1-results}
  \end{center}
\end{figure}


% \subsection{Advection and Diffusion through a Multiple Links}
% \label{sec:test-advection-diffusion-2-link}

% \begin{figure}[htbp]
%   \begin{center}
%     \includegraphics[width=4.5in]{../transport/advection-diffusion-2-segment/plot-Cn=1.0.eps}
%     \includegraphics[width=4.5in]{../transport/advection-diffusion-2-segment/plot-Cn=0.1.eps}
%     \caption{Results for the test of advective diffusion through multiple links;
%       simulated concentration is compared to an analytical solution.}
%     \label{fig:test-advection-diffusion-2-results}
%   \end{center}
% \end{figure}

% \clearpage{}


\subsection{Transport with Lateral Inflow/Outflow}
\label{sec:test-transport-lateral}

In the current MASS1 version, lateral inflow is considered to have the
same gas concentration and temperature as the water in the main
channel.  This will probably change, so that concentrations and
temperatures of lateral inflow can be specified.  This test is used to
verify that when there is lateral inflow/outflow, it does not affect
the simulated channel concentrations.  

Using the channel configuration of
Section~\ref{sec:test-advection-1-link}, a lateral inflow of 500 cfs
was applied along the entire length of channel ($4.692 \times 10^{-2}
cfs/ft$), and a constant inflow concentration (10.0) was applied
upstream.  This was allowed to come to steady state, with the
expectation that the concentration would remain at 10.0 throughout the
channel.  The results are shown in
Figure~\ref{fig:test-tranport-lateral-inflow}, which show that the
concentration did indeed remain constant. 

A similar simulation was performed with a lateral \textit{outflow} of
500 cfs applied along the entire length of channel.  The results of
this simulation are shown in
Figure~\ref{fig:test-tranport-lateral-outflow}.

\begin{figure}[htbp]
  \begin{center}
    \includegraphics[width=5in]{../transport/lateral-inflow/plot-inflow.eps}
    \caption{Results for test of transport with lateral inflow;
      concentration remains constant because the concentration of
      lateral inflow is assumed to be that in the channel.} 
    \label{fig:test-tranport-lateral-inflow}
  \end{center}
\end{figure}

\begin{figure}[htbp]
  \begin{center}
    \includegraphics[width=4.5in]{../transport/lateral-inflow/plot-outflow.eps}
    \caption{Results for test of transport with lateral outflow.} 
    \label{fig:test-tranport-lateral-outflow}
  \end{center}
\end{figure}

% \begin{figure}[htbp]
%   \begin{center}
%     \includegraphics[width=4.5in]{../transport/lateral-inflow/bcplot.eps}
%     \caption{Total lateral inflow used to test transport with varying
%       lateral contribution.} 
%     \label{fig:test-transport-lateral-flow}
%   \end{center}
% \end{figure}

% \begin{figure}[htbp]
%   \begin{center}
%     \includegraphics[width=4.5in]{../transport/lateral-inflow/plot-vary.eps}
%     \caption{Results for test of transport with varying lateral
%       contribution.}  
%     \label{fig:test-tranport-lateral-vary}
%   \end{center}
% \end{figure}


\end{document}


%%% Local Variables: 
%%% mode: latex
%%% TeX-master: t
%%% End: 
